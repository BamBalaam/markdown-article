\chapter{Introduction}

\vspace{1cm}

Created by John Gruber in 2004, with extensive help by the late Aaron Swartz, Markdown is a markup language that became extremely popular
and widely adopted since then. It allows its users to format text using a syntax which focuses on readability and simplicity.\footcite["Syntax" page]{gruber2004markdown}\\

Drawing inspiration from previously existing markup languages that were used on email and usenet clients, such as reStructuredText and Textile,
and inspired by Aaron Swartz's recently created markup language named atx, John Gruber created two things: the syntax for a language, and a simple
Perl script ("Markdown.pl") which replaces the markers for that language with HTML tags.\footcite{gruber2004markdown}\\

%TODO:
% - Focus on the human-readability of the syntax, with examples and comparisons (RTF, HTML, wikitext)

This article aims to explore the origins of Markdown and markup languages in general, the role and influence of Markdown in various fields,

This article aims to explore the Markdown standard (or lack thereof), explore why several variations of it exist, and address the issues
arising from the diversity of these variations.