\chapter{Conclusion}
\label{chap:conclusion}

\vspace{1cm}

Markdown drew from a long history of markup languages and influenced several other after its conception. Its impact as a piece of
technology is a fact that few people could disprove: how such a simple concept evolved into a massive set of tools and implementations
is a testament to its success, and the fact that it covers such a broad set of use-cases too.\newline

Seen by many as a flaw in the original design, its lack of standardization was probably what made it so adaptable in the first
place, and why so many people took it upon themselves to build over its simple base. This article's author posits that were it
very rigid to begin with, it might not have had the same impact in the long run.\newline

A rigid standard could be a double edged sword: it could, of course, limit a lot of the issues we see in terms of diverging HTML
outputs from implementations in the wild, and it would help people learn its syntax in a easier way. However, given that most
implementations create supersets of the syntax to augment the features in their tools, users would still encounter issues migrating
their documents from one tool to the other, no matter the underlying core syntax that exists. This probably means that no ``true standard''
will ever really exist, no matter the efforts.