\chapter{Markdown ``flavours'', extensions, and their impacts}
\label{chap:proliferation}

\vspace{1cm}

As seen in chapter \ref{chap:issues}, the qualities of Markdown outweighing its flaws, and opinionated people wanting to fix issues
their own way, the proliferation of new syntaxes and interpreters that are supersets or modifications of the original syntax made their appearance,
which are commonly called ``markdown flavours''. These allowed many tools and use-cases to come to life, which go far beyond than simple document
markup.

\cite{voegler2014markdown}