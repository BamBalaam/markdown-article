\chapter{Issues surrounding Markdown}
\label{chap:issues}

\vspace{1cm}

Markdown achieved its objectives of readability and simplicity, which were the main arguments for its fast rise in popularity.
What started as a simple Perl script that converts Markdown to HTML via regex replacements became a phenomenon, and was adopted in many
famous online services and pieces of software. However, there is no real ``specification'' for Markdown beyond the original blog post
introducing the language and a few other pages on Gruber's website.\footcite{gruber2004markdown}
With time, one of Markdown's strengths, it's simplicity, also became one of its flaws for a few reasons:

\begin{itemize}
    \item While integrating Markdown, many developers realised that unfortunately the official Perl script is filled with unintentional
    blind spots and breaking states, which generate invalid HTML;
    \item Users felt that some necessary features were missing;
    \item Other output formats than HTML were often requested, most notably PDF.
\end{itemize}

These issues led to various implementations and syntaxes appearing all over the web-space (see chapter \ref{chap:proliferation}),
most of them unfortunately incompatible with each other.

\section{Blind spots}

Creators of alternative Markdown implementations list several blind spots and issues in the original Markdown specification in order to
justify their creations. Listed below is a non-exhaustive list of the most reoccurring ones:\newline

\textbf{The number of space characters required in order to distinguish elements of a sub-list in a list}\newline

xxx\newline

\textbf{Whether blank lines are required or not after a block quote or header}\newline

xxx\newline

\textbf{Is a space character allowed or not when creating inline links}\newline

xxx\newline

\textbf{}\newline

xxx\newline

\textbf{}\newline

xxx\newline

\textbf{}\newline

xxx\newline

\newpage

\section{Missing features}

Users of Markdown gradually felt that its simplicity was a crutch, given that they really wanted to used this syntax language but several
features were missing, in their opinion. In chapter \ref{chap:proliferation}, we'll address how these missing features were addressed, but
in the meantime here is a list of reoccurring issues users reported:

\begin{itemize}
    \item Limited formatting options (lack of tables, footnotes, task lists, syntax highlighting for code blocks, and others)
    \item Lack of accessibility (complex documents tend to be tough to structure for blind users for example);
    \item No support for macros or variables, although the original spec outlines that ``curly braces'' (the characters ``\{'' and ``\}''
    are left as unused markers for that usage);
    \item Lack of metadata capabilities.
\end{itemize}

\section{Output formats}

As mentioned above, many users were interested in getting something else than HTML out of Markdown. At its core, Markdown was thought as
a language for the web, as a minified way to see HTML, but as users started to use Markdown for other things such as code documentation,
HTML was deemed too restrictive of a format. For this use-case, for example, they would prefer a more ``presentable'' format to be stored
for posterity, or for dissemination to colleagues and management, such as PDF or DOC.\newline

Various projects emerged in order to do so, the most famous of which being Pandoc\footcite{pandoc}.

\footcite{dominici2014overview}